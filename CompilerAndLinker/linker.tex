\documentclass[english]{lni}
\usepackage{listings}
\usepackage{xcolor}

\author{Alfred Mikhael}
\title{C++ Linker}

\definecolor{background}{rgb}{0.95, 0.95, 0.95}

\lstdefinestyle{myStyle}{
    backgroundcolor = \color{background},
    keywordstyle = \color{blue},
    numberstyle = \color{darkgray},
    stringstyle = \color{teal},
    identifierstyle = \color{black}
}

\lstset{style = myStyle}

\begin{document}

\section{Linking in C++}
Each translation unit (.ccp file) has no relation to other translation units.
Because of this, if we want to split our program into multiple files, we need
a way to link them together into one executable file.

Even if we don't use multiple files, we need to link functions to each other
so that we know the entry point (where to start the instructions).

\subsection{When does linking occur}
Linking occurs whenever a function may be called. Errors regarding function linking
are different from compilation errors, which occur when there is no function
declaration available.

In the following example, the project will build successfully although you might
think it would raise a linking error.
\begin{lstlisting}[language = C++]
    \\ Log.cpp
    void Logr(const char* message) {
        return message;
    }

    \\ Main.cpp
    void Log(const char* message);

    static int multiply(int a, int b) {
        Log("message");
        return a * b;
    }

    int main() {
        return 0;
    }
\end{lstlisting}
This does not raise an error because \verb$multiply$ is a static function, which means
that it can only be accessed from inside \verb$main.cpp$. Since it is not called inside
of the main function, the compiler ignores it. If \verb$multiply$ is not marked as a
static function, there will be a linking error, since it could be called from other cpp files.

\subsection{Ambiguous symbols}
Another common linking error can occur when multiple functions are defined with the same
signature. This may seem like an uncommon error, however, look at this example.

\begin{lstlisting}[language = C++]
    \\ log.h
    #pragma once
    #include <iostream>

    void log(const char* message) {
        std::cout << message << std::endl;
    }

    \\ initLog.cpp
    #include "log.h"

    void initLog() {
        log("Initialized");
    }

    \\ main.cpp 
    #include "log.h"

    int main() {
        log("message");
        return 0;
    }
\end{lstlisting}

In this example, the contents of \verb$log.h$ are copied into both \verb$initLog.cpp$
and \verb$main.cpp$. This means that there are two identical functions, which
causes a linking error. This is why you should \textbf{never} put function definitions inside
of a header file!

This error can be avoided by identifying \verb$log$ as a static function, so
that the function in \verb$initLog.cpp$ has a different signature from the function in
\verb$main.cpp$, by marking \verb$log$ as inline, or by moving the definition into a .cpp file
and leaving only the declaration in the header.

\subsection{Errors}
Since compilation and linking are separate processes, compilation and linking
will give us separate errors. It is very important to know the difference
when debugging your program. The error code of compilation errors will start
with a C \ie \emph{C0239}.

\end{document}